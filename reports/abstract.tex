% !TeX spellcheck = en_US
\chapterstyle{thesissimplechapter}


% This is a hack...
\chapter*{}

\vspace{-1cm}

%\chapter*{Main goals}
\begin{center}
	\fontsize{\chaptitlefontsize}{0}\selectfont \textbf{Main goals}
\end{center}

The present protocol contains the theories that support my doctoral research in the field of quantum many-body fluids
as well as a set of preliminary results which I have obtained during the last three academic semesters.

The main objective of my investigation is to study the physics of quantum gases constrained within periodic structures
like planar, multi-slabs structures. In order to estimate the ground state properties of such systems as well as
identify and characterize their possible quantum phase transitions I use mean-field approximations and
---with a major emphasis--- Quantum Monte Carlo methods.

My investigation is directed to acquire and dominate one of the most robust tools to study quantum many-particles
systems: the Diffusion Monte Carlo method, which approximates the ground state energy of an interacting Bose gas up to
a manageable statistical error. We will apply this method to calculate the ground state energy, the momentum
distribution and the static structure factor.


\vspace{1cm}


\begin{center}
	\fontsize{\chaptitlefontsize}{0}\selectfont \textbf{Objetivos principales}
\end{center}

\hyphenation{pre-li-mi-na-res}
\hyphenation{pla-na-res}

Este protocolo resume las teorías que utilizo en mi investigación doctoral en el campo de los sistemas
cuánticos de muchos cuerpos, así como los resultados preliminares obtenidos durante los últimos tres semestres
académicos.

El objetivo principal de mi investigación es el estudio de la física de gases cuánticos constreñidos dentro de
estructuras periódicas tales como estructuras multi-losas o planares. A fin de estimar las propiedades del estado base
de tales sistemas así como para identificar y caracterizar sus posibles transiciones de fase cuánticas empleo
aproximaciones de campo medio y ---con mayor énfasis--- métodos de Monte Carlo.

My investigación está encaminada a adquirir y dominar una de las herramientas más robustas para estudiar gases
cuánticos: el método de Monte Carlo Difusivo (Diffusion Monte Carlo), el cual aproxima
la energía del estado base de un gas de Bose salvo un error estadístico controlable. Aplicaremos este método para
calcular la energía del estado base, la distribución de momentos y el factor de estructura estática.


% Reset the chapter style
\chapterstyle{thesischapter}
