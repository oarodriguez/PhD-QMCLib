
% !TeX spellcheck = en_US
\chapter{Introduction}
\label{sec:introduccion}

Ultracold bosonic and fermionic quantum gases have been a subject of intense
research in the last few years since the experimental realization of
Bose-Einstein condensation in 1995 \cite{bib:anderson-science.269.1995,
bib:davis-phys.rev.A.75.3969.1995}. These experiments were a milestone that
transformed Bose and Einstein \cite{bib:einstein-1924-quantentheorie}
mathematical predictions in a physical reality. Experimental techniques have
evolved enormously and physicists are able to realize Bose-Einstein condensates
with great control of the particle interaction and confined in diverse
geometries. These developments have opened the possibility to experiment with
novel quantum many-body systems \cite{bib:takasu-et-al-phys-rev-lett.91.040404,
bib:schreck-f-et-al-phys-rev-lett.87.080403}: Fermi and Bose gases are
constrained in such ways that their movement is effectively quenched in one or
two dimensions, so the dimensionality of the systems is not longer
three-dimensional \cite{bib:a-gorlitz-phys-rev-lett.87.130402}. And not only the
geometry of the systems can be controlled: other parameters as the interaction
strength between the particles, the density, the temperature and the spin can be
adjusted. We have reached the point where experimental researchers are able to
simulate \cite{bib:bloch-nature-physics.2012} real quantum many-body systems in
the laboratory by controlling most of its variables at will to the point where
theoretical predictions made long ago are being verified
\cite{bib:tonks-phys-rev.50.955.1936, bib:girardeau-j-math-phys.1960,
bib:paredes-bloch-et-al-nature.419.2014}, exotic phase transitions are being
discovered and new predictions are being done.

Among the quantum systems that can be simulated with ultracold gases, periodic
ones arise naturally as a consequence of the techniques used to constrain the
particles, particularly when the confining potentials are made from optical
lattices \cite{bib:bloch-nature-phys.2005}. It is worth to say that the shape of
the potentials that trap the bosons or fermions can be controlled as well. This
is a significant fact, as many of the most important problems in condensed
matter physics such as superconductivity
\cite{bib:bednorz-et-al-phys-b-cond-matt.1986, bib:phys-rev-B.50.4260,
bib:norman-pepin-rep-prog-phys.2003} and helium-4 in any dimension
\cite{bib:phys-rev-lett.71.3673, bib:j-phys-soc-jpn.77.111011} involve physical
periodic systems.

From the theoretical aspect, we know that the properties of real quantum systems
are a consequence of the combined effect of the interactions between the
particles and the spatial constrictions. In order to study quantum systems we
have to take into account the interactions between its particles. This is not an
easy task by any means, as we have to solve the Schrödinger equation of $N$
particles! This means to find the corresponding wave function of the system
(symmetric or antisymmetric depending on whether the particles are bosons or
fermions) in a $3N$ dimensional coordinate space (or momentum space) so we have
the possibility to obtain expected values of the observables of interest, as the
momentum distribution. The direct solution of the Schrödinger equation is an
unfeasible task unless the system under study contains only a few particles.

Our research is focused mainly on bosons. The most common theoretical tool to
study bosonic quantum many-body systems is mean-field theory, i.e., the
Gross-Pitaevskii theory \cite{bib:gross-il-nuovo-cimento.1961,
bib:pitaevskii-jetp.1961}, This approach assumes that the vast majority of the
bosons occupy the same quantum state. This state (the ground state) is described
by a wave function whose evolution in time is determined by the Gross-Pitaevskii
equation: a one-body Schrödinger equation that takes the interaction between the
bosons into account in the form of an additional nonlinear term. This theory has
become a very useful tool as an alternative to the direct solution of the
$N$-body Schrödinger equation. However, the applicability of the GP theory is
restricted to systems where the healing length (or coherence length) is much
larger than the distance between the bosons
\cite{bib:pitaevskii-rev-mod-phys.71.463.1999}. For strongly correlated quantum
systems this is not true anymore.

Quantum Monte Carlo ({\textbf{QMC}}) methods have become an attractive
alternative to calculate the properties of the ground state of quantum many-body
systems at zero or even finite temperature
\cite{bib:ceperley-rev-min-geochem.1.2010}. They give us information of a
quantum system without having to solve the Schrödinger equation, but instead we
have to evaluate multidimensional integrals for the expected values of the
observables of interest, being the energy one of the most important, although
other properties such as the momentum distribution and the static structure
factor can be estimated. The main problem that Monte Carlo techniques address
successfully is the evaluation of multidimensional integrals that appear when
evaluating the expected values of some observable, since for a system with $N$
particles we have to integrate over a $3N$-dimensional space.

Most of the numerical techniques to evaluate integrals in one, two or three
dimensions suffer of a problem called \textit{the curse of the dimensionality},
as the number of function evaluations to reach a given accuracy increases
exponentially with the dimension of the space. Monte Carlo methods are
stochastic algorithms that do not have this problem, as its rate of convergence
is inversely proportional to the square root of the number of evaluations.
Although it has a slow rate of convergence, it is independent of the dimension
of the space. In Quantum Monte Carlo, the number of particles determines the
size of the coordinate space, but the convergence rate of the algorithm will be
the same.

Among the various QMC methods, the Variational Monte Carlo approach
({\textbf{VMC}}) is the most simple. It is based on the variational principle of
Quantum Mechanics, so we can use it to obtain an upper bound estimate to the
ground state of the system. How good this estimate is depends on the quality of
the trial wave function used to evaluate the expected value of the observable in
question. Another approach is the Diffusion Monte Carlo method (\textbf{DMC}),
an algorithm that starts from an initial approximation of the wave function and
iteratively solves the \textit{imaginary time Schrödinger equation}. This
process is repeated until we have a good enough estimate of the exact ground
state wave function of the system. This means that any calculated expected value
will be exact, except for a small, manageable stochastic noise. The Diffusion
Monte Carlo method, however, is a lot more complex than the Variational Monte
Carlo approach. Both techniques have been used extensively to study strongly
correlated systems
\cite{bib:boronat-phys-rev-B.49.8920.1994,bib:casurellas-phys-rev-B.52.3654.1995,
bib:astrakharchik-phys-rev-a.68.031602.2003,
bib:astrakharchik-phys.rev.lett.95.190407.2005} with great success.

The main objective of my investigation is to study the physics of quantum gases
constrained within periodic structures like planar, multi-slabs structures. In
order to estimate the ground state properties of such systems as well as
identify and characterize their possible quantum phase transitions, I use
several methods, but I focus mainly Quantum Monte Carlo methods.

This protocol is organized in the following way:

In chapter \ref{chap:background} we give an brief overview of the research we
did on the physical properties of an ideal Bose gas within a multi-slabs
structure. We obtained the energy spectrum and the Bose-Einstein condensation
critical temperature.
%In addition we studied the thermodynamic properties like the chemical
%potential, the heat capacity both at constant volume and constant pressure and
%both the isothermal and isentropic (adiabatic) sound velocities.
In addition we describe the equation of state of a Lieb-Liniger gas, as this
system is a natural reference.

Chapter \ref{chap:mean-field-theory} contains the theory of diluted gases in
three dimensions, also known as mean-field theory. In this chapter we give some
general results for the energy of a weakly interacting Bose gas and develop the
mathematics that lead us to the celebrated Gross-Pitaevskii equation for a gas
within an inhomogeneous medium. Later we explain how this theory must be
adjusted in order to be applicable to Bose gases in lower dimensions,
particularly for systems with very elongated, cigar-like shapes that resemble
one-dimension geometry. We establish the conditions under which the mean-field
theory picture is applicable to one-dimensional systems, to finally arrive to
the one-dimensional version of the Gross-Pitaevskii equation.

The most important section of this chapter contains the procedure to arrive to
the solution of the Gross-Pitaevskii equation in 1D for a Bose gas subject to
Kronig-Penney potential, which has an analytical form given in terms of the
Jacobi elliptic functions. We establish the boundary conditions corresponding to
periodic solutions which must be applied to obtain the density profile of the
gas and other physical properties such as the chemical potential and the energy.
Then we show some preliminary numerical results for a gas with repulsive
interactions among its components as well as for a gas with attractive
interactions, and show the existence of stable solutions for this case.


In Chapter \ref{chap:quantum-monte-carlo-methods} we give a description of the
Quantum Monte Carlo methods and how are they used to evaluate multidimensional
integrals like those that arise when evaluating expected value of observables in
quantum mechanics. Then we describe the foundation of such algorithms, from the
concept of a Markov chain to the description and application of the
Metropolis-Hastings algorithm to sample a complicated probability distribution
function. We also show how they are used to obtain the physical properties of
the ground state of a many-body quantum system with \textit{ab-initio}
interactions, i.e., with a real, microscopic interaction potential.

We focus on the Variational Monte Carlo method. As in the case of mean-field
theory, we present some preliminary results of the energy for an interacting
Bose gas in 1D subject to a Kronig-Penney potential using this quantum Monte
Carlo technique. We use two different variational functions to estimate the
energy, and later we investigate how the variational parameters affect the
energy behavior, and which values minimize the energy. Then we compare both
models and determine which is the best. Finally we analyze these results
together with the predictions of the Lieb-Liniger theory so as with the results
calculated from mean-field theory. We do this in order to validate our results
as well as be confident with further results to come in the future.

We finish this protocol giving an outline of my future research as well as a
schedule of my research work for the next two years.
